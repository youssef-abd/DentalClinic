\documentclass[12pt,a4paper]{report}
\usepackage[utf8]{inputenc}
\usepackage[T1]{fontenc}
\usepackage[french]{babel}
\usepackage{graphicx}
\usepackage{geometry}
\usepackage{setspace}
\usepackage{titlesec}
\usepackage{fancyhdr}
\usepackage{hyperref}
\usepackage{booktabs}
\usepackage{tabularx}
\usepackage{enumitem}

% Configuration de la page
\geometry{hmargin=2.5cm, vmargin=2.5cm}
\onehalfspacing

% Style des titres
\titleformat{\chapter}[display]
  {\normalfont\huge\bfseries}{\chaptertitlename\ \thechapter}{20pt}{\Huge}
\titlespacing*{\chapter}{0pt}{-30pt}{40pt}

% En-têtes et pieds de page
\pagestyle{fancy}
\fancyhf{}
\fancyhead[L]{\leftmark}
\fancyfoot[C]{\thepage}
\renewcommand{\headrulewidth}{0.4pt}

% Informations du document
\title{\textbf{Rapport de Stage}\\[0.5cm]
Développement d'une Application de Gestion pour Cabinet Dentaire}
\author{Votre Nom}
\date{Année Universitaire 2024-2025}

\begin{document}

\begin{titlepage}
    \centering
    \vspace*{2cm}
    \includegraphics[width=0.3\textwidth]{logo_emsi.png}\par\vspace{1cm}
    {\LARGE École Marocaine des Sciences de l'Ingénieur\par}
    \vspace{2cm}
    {\Huge\bfseries Rapport de Stage\par}
    \vspace{0.5cm}
    {\Large\bfseries Développement d'une Application de Gestion pour Cabinet Dentaire\par}
    \vspace{2cm}
    {\large \textbf{Étudiant:} Votre Nom\par}
    {\large \textbf{Encadrant Pédagogique:} Nom de l'encadrant\par}
    {\large \textbf{Maître de Stage:} Dr. Nom du Docteur\par}
    {\large \textbf{Durée:} Du JJ/MM/AAAA au JJ/MM/AAAA\par}
    \vfill
    {\large Année Universitaire 2024-2025\par}
\end{titlepage}

\tableofcontents
\thispagestyle{empty}
\cleardoublepage

\setcounter{page}{1}

\chapter*{Remerciements}
\addcontentsline{toc}{chapter}{Remerciements}

Je tiens à exprimer ma profonde gratitude à toutes les personnes qui ont contribué de près ou de loin à la réalisation de ce stage et à l'élaboration de ce rapport.\par

Un remerciement tout particulier au Dr. [Nom] pour m'avoir accordé sa confiance et m'avoir offert cette opportunité d'apprentissage exceptionnelle au sein de son cabinet dentaire.\par

Je remercie également [Nom de l'encadrant pédagogique] pour son encadrement, ses conseils avisés et sa disponibilité tout au long de ce stage.\par

Enfin, je tiens à remercier toute l'équipe du cabinet pour leur accueil chaleureux, leur soutien et leur collaboration qui ont grandement contribué à la réussite de ce projet.

\chapter{Introduction}
\section{Présentation du cabinet dentaire}
Le cabinet dentaire [Nom du Cabinet], situé à [Ville], est une structure de soins dentaires offrant des services complets en dentisterie générale et spécialisée. Fondé en [année], le cabinet accueille une équipe de [nombre] praticiens et [nombre] assistantes dentaires, prenant en charge en moyenne [nombre] patients par mois.\par

L'établissement se distingue par son approche patient-centrée et son engagement envers les technologies de pointe en matière de soins dentaires.

\section{Contexte du stage}
Ce stage s'inscrit dans le cadre de ma formation d'ingénieur en génie logiciel à l'EMSI. Le choix de ce stage s'est imposé par mon intérêt pour le développement d'applications métiers et mon désir de contribuer à l'amélioration des processus de gestion dans le domaine de la santé.\par

Les objectifs principaux de ce stage étaient :
\begin{itemize}
    \item Comprendre les besoins spécifiques de gestion d'un cabinet dentaire
    \item Concevoir et développer une solution logicielle sur mesure
    \item Mettre en œuvre des bonnes pratiques de développement logiciel
    \item Assurer une formation efficace du personnel
\end{itemize}

\section{Présentation du projet}
Le projet consiste en le développement d'une application de gestion complète pour le cabinet dentaire, comprenant les modules suivants :
\begin{itemize}
    \item Gestion des patients et dossiers médicaux
    \item Prise de rendez-vous et planning
    \item Gestion du stock et des fournitures dentaires
    \item Facturation et suivi des paiements
    \item Tableaux de bord et statistiques
\end{itemize}

\chapter{Analyse des besoins}
\section{Méthodologie d'analyse}
L'analyse des besoins a été réalisée selon une approche itérative et participative, combinant :
\begin{itemize}
    \item Des entretiens semi-directifs avec le Dr. [Nom] et le personnel
    \item L'observation des processus existants
    \item L'analyse des documents de gestion actuels
    \item Des ateliers de conception participative
\end{itemize}

\section{Besoins identifiés}
\subsection{Besoins fonctionnels}
\begin{itemize}
    \item Gestion centralisée des dossiers patients
    \item Planification des rendez-vous avec gestion des conflits
    \item Suivi des traitements et prescriptions
    \item Gestion du stock avec alertes de réapprovisionnement
    \item Édition de factures et suivi des paiements
    \item Tableaux de bord de suivi d'activité
\end{itemize}

\subsection{Besoins non-fonctionnels}
\begin{itemize}
    \item Interface intuitive et adaptée aux contraintes du cabinet
    \item Performance : Temps de réponse < 2 secondes
    \item Sécurité : Conformité RGPD, gestion des habilitations
    \item Disponibilité : 99.9\% en heures d'ouverture
    \item Portabilité : Compatible Windows 10/11
\end{itemize}

\section{Priorisation des besoins}
La priorisation a été établie selon la méthode MoSCoW :
\begin{itemize}
    \item \textbf{Must have} : Gestion patients, rendez-vous, facturation
    \item \textbf{Should have} : Gestion du stock, statistiques
    \item \textbf{Could have} : Interface mobile, notifications SMS
    \item \textbf{Won't have} : Gestion de la paie du personnel
\end{itemize}

\chapter{Conception de la solution}
\section{Choix technologiques}
\begin{itemize}
    \item \textbf{Frontend} : PyQt6 pour une application bureau native
    \item \textbf{Backend} : Python 3.10+
    \item \textbf{Base de données} : SQLite pour la portabilité
    \item \textbf{Contrôle de version} : Git avec GitHub
    \item \textbf{Gestion des dépendances} : pip avec requirements.txt
\end{itemize}

\section{Architecture du système}
L'application suit une architecture MVC (Modèle-Vue-Contrôleur) avec :
\begin{itemize}
    \item \textbf{Modèle} : Gestion des données et logique métier
    \item \textbf{Vue} : Interface utilisateur PyQt6
    \item \textbf{Contrôleur} : Gestion des interactions utilisateur
\end{itemize}

\section{Modèle de données}
Le schéma de la base de données comprend les entités principales :
\begin{itemize}
    \item Patients
    \item Rendez-vous
    \item Traitements
    \item Stock
    \item Factures
    \item Utilisateurs
\end{itemize}

\chapter{Développement}
\section{Organisation du travail}
Le développement a été organisé en sprints de 2 semaines avec :
\begin{itemize}
    \item Réunions de planification en début de sprint
    \item Démonstrations hebdomadaires
    \item Rétrospectives pour amélioration continue
\end{itemize}

\section{Modules développés}
\subsection{Gestion des patients}
\begin{itemize}
    \item Création et édition des dossiers patients
    \item Historique médical complet
    \item Gestion des documents et radiographies
\end{itemize}

\subsection{Gestion des rendez-vous}
\begin{itemize}
    \item Planification avec vue jour/semaine/mois
    \item Gestion des rappels
    \item Gestion des annulations et reports
\end{itemize}

\subsection{Gestion du stock}
\begin{itemize}
    \item Suivi des produits et fournitures
    \item Alertes de seuil minimum
    \item Gestion des commandes fournisseurs
\end{itemize}

\subsection{Facturation}
\begin{itemize}
    \item Création de devis et factures
    \item Suivi des paiements
    \item Génération d'états financiers
\end{itemize}

\chapter{Résultats et bénéfices}
\section{Fonctionnalités livrées}
L'application comprend actuellement :
\begin{itemize}
    \item Gestion complète des patients et dossiers
    \item Planification des rendez-vous
    \item Gestion du stock
    \item Module de facturation
    \item Tableaux de bord
\end{itemize}

\section{Bénéfices pour le cabinet}
\begin{itemize}
    \item Réduction du temps de gestion administrative
    \item Meilleure organisation des rendez-vous
    \item Optimisation de la gestion des stocks
    \item Amélioration du suivi financier
\end{itemize}

\chapter{Perspectives d'évolution}
\section{Améliorations prévues}
\begin{itemize}
    \item Application mobile pour la gestion à distance
    \item Intégration avec des logiciels de comptabilité
    \item Module de télémédecine
    \item Analyse prédictive pour la gestion des stocks
\end{itemize}

\chapter{Conclusion}
Ce stage a été une expérience extrêmement enrichissante qui m'a permis de :
\begin{itemize}
    \item Développer mes compétences techniques en développement logiciel
    \item Comprendre les enjeux spécifiques du secteur de la santé
    \item Améliorer mes compétences en gestion de projet
    \item Travailler en équipe avec des professionnels de santé
\end{itemize}

\appendix
\chapter{Captures d'écran}
\chapter{Manuel d'utilisation}
\chapter{Code source}

\end{document}
